\subsection{Módulo 2: Fog Computing - TTN (The Things Network)}


\subsubsection{Gestión de Downlinks}


\begin{table}[H]
\centering
\begin{tabular}{|l|p{10cm}|}
\hline
\textbf{Identificador} & PU-M2-05 \\ \hline
\textbf{Descripción} & Verificar la codificación correcta de comandos de control \\ \hline
\textbf{Precondiciones} & Payload formatter configurado para downlinks \\ \hline
\textbf{Pasos a seguir} & 
1. Configurar encoder en TTN para comandos ON/OFF \\
2. Enviar comando "ON" (debe codificarse como 0x01) \\
3. Enviar comando "OFF" (debe codificarse como 0x00) \\
4. Verificar codificación en Live Data de TTN \\ \hline 
\textbf{Resultado esperado} & TTN debe codificar correctamente: "ON" → 0x01 y "OFF" → 0x00 en el payload del downlink. \\ \hline
\end{tabular}
\caption{Caso de prueba PU-M2-05: Verificación de codificación de downlinks}
\label{tab:pu-m2-05}
\end{table}

\subsection{Módulo 3.1: Cloud Computing - Servidor Web (Backend)}

\subsubsection{APIs de Autenticación}

\begin{table}[H]
\centering
\begin{tabular}{|l|p{10cm}|}
\hline
\textbf{Identificador} & PU-M3.1-01 \\ \hline
\textbf{Descripción} & Verificar el funcionamiento de la API de registro de usuarios \\ \hline
\textbf{Precondiciones} & Servidor backend funcionando, base de datos accesible \\ \hline
\textbf{Pasos a seguir} & 
1. Enviar POST /api/auth/register con datos válidos \\
2. Verificar creación de usuario en base de datos \\
3. Comprobar envío de email de verificación \\
4. Probar registro con email duplicado \\ \hline 
\textbf{Resultado esperado} & Debe retornar 201 para registro exitoso, crear usuario no verificado, enviar email y retornar 400 para email duplicado. \\ \hline
\end{tabular}
\caption{Caso de prueba PU-M3.1-01: Verificación de API de registro}
\label{tab:pu-m3.1-01}
\end{table>

\begin{table}[H]
\centering
\begin{tabular}{|l|p{10cm}|}
\hline
\textbf{Identificador} & PU-M3.1-02 \\ \hline
\textbf{Descripción} & Verificar el funcionamiento de la API de login \\ \hline
\textbf{Precondiciones} & Usuario verificado existente en base de datos \\ \hline
\textbf{Pasos a seguir} & 
1. Enviar POST /api/auth/login con credenciales válidas \\
2. Verificar generación de token JWT \\
3. Probar login con credenciales incorrectas \\
4. Probar login con usuario no verificado \\ \hline 
\textbf{Resultado esperado} & Debe retornar 200 con token JWT válido para credenciales correctas, 401 para incorrectas y 403 para no verificados. \\ \hline
\end{tabular}
\caption{Caso de prueba PU-M3.1-02: Verificación de API de login}
\label{tab:pu-m3.1-02}
\end{table>

\subsubsection{APIs de Gestión de Dispositivos}

\begin{table}[H]
\centering
\begin{tabular}{|l|p{10cm}|}
\hline
\textbf{Identificador} & PU-M3.1-03 \\ \hline
\textbf{Descripción} & Verificar CRUD completo de dispositivos IoT \\ \hline
\textbf{Precondiciones} & Usuario autenticado con token JWT válido \\ \hline
\textbf{Pasos a seguir} & 
1. POST /api/devices - crear dispositivo con credenciales TTN \\
2. GET /api/devices - listar dispositivos del usuario \\
3. PUT /api/devices/:id - actualizar configuración \\
4. DELETE /api/devices/:id - eliminar dispositivo \\ \hline 
\textbf{Resultado esperado} & Todas las operaciones CRUD deben funcionar correctamente con validación de autorización y datos TTN válidos. \\ \hline
\end{tabular}
\caption{Caso de prueba PU-M3.1-03: Verificación de APIs de dispositivos}
\label{tab:pu-m3.1-03}
\end{table>

\subsubsection{APIs de Gestión de Cultivos}

\begin{table}[H]
\centering
\begin{tabular}{|l|p{10cm}|}
\hline
\textbf{Identificador} & PU-M3.1-04 \\ \hline
\textbf{Descripción} & Verificar CRUD y selección de cultivos \\ \hline
\textbf{Precondiciones} & Usuario autenticado \\ \hline
\textbf{Pasos a seguir} & 
1. POST /api/crops - crear cultivo con parámetros completos \\
2. GET /api/crops - listar cultivos del usuario \\
3. PUT /api/crops/:id/select - seleccionar cultivo activo \\
4. Verificar que solo un cultivo esté activo a la vez \\ \hline 
\textbf{Resultado esperado} & CRUD de cultivos debe funcionar correctamente y la selección debe permitir solo un cultivo activo por usuario. \\ \hline
\end{tabular}
\caption{Caso de prueba PU-M3.1-04: Verificación de APIs de cultivos}
\label{tab:pu-m3.1-04}
\end{table>

\subsubsection{APIs de Control de Riego}

\begin{table}[H]
\centering
\begin{tabular}{|l|p{10cm}|}
\hline
\textbf{Identificador} & PU-M3.1-05 \\ \hline
\textbf{Descripción} & Verificar APIs de control de riego y downlinks TTN \\ \hline
\textbf{Precondiciones} & Dispositivo activo, cultivo seleccionado \\ \hline
\textbf{Pasos a seguir} & 
1. POST /api/irrigation/start - iniciar riego manual \\
2. Verificar envío de downlink 0x01 a TTN \\
3. POST /api/irrigation/stop - detener riego \\
4. Verificar envío de downlink 0x00 a TTN \\ \hline 
\textbf{Resultado esperado} & APIs deben enviar correctamente downlinks a TTN y actualizar estado de riego en base de datos. \\ \hline
\end{tabular}
\caption{Caso de prueba PU-M3.1-05: Verificación de APIs de riego}
\label{tab:pu-m3.1-05}
\end{table>

\subsubsection{API de Webhook TTN}

\begin{table}[H]
\centering
\begin{tabular}{|l|p{10cm}|}
\hline
\textbf{Identificador} & PU-M3.1-06 \\ \hline
\textbf{Descripción} & Verificar recepción y procesamiento de datos del webhook TTN \\ \hline
\textbf{Precondiciones} & Webhook configurado en TTN apuntando al backend \\ \hline
\textbf{Pasos a seguir} & 
1. Simular POST desde TTN con payload de sensores \\
2. Verificar validación de DevEUI autorizado \\
3. Comprobar almacenamiento en tabla SensorReading \\
4. Verificar que solo dispositivos activos guarden datos \\ \hline 
\textbf{Resultado esperado} & Backend debe validar dispositivo, procesar datos y almacenar solo si comunicación está activa. \\ \hline
\end{tabular}
\caption{Caso de prueba PU-M3.1-06: Verificación de API webhook TTN}
\label{tab:pu-m3.1-06}
\end{table}

\subsection{Módulo 3.2: Cloud Computing - Base de Datos}

\begin{table}[H]
\centering
\begin{tabular}{|l|p{10cm}|}
\hline
\textbf{Identificador} & PU-M3.2-01 \\ \hline
\textbf{Descripción} & Verificar integridad referencial y constraints de base de datos \\ \hline
\textbf{Precondiciones} & Base de datos PostgreSQL configurada con esquema completo \\ \hline
\textbf{Pasos a seguir} & 
1. Intentar insertar SensorReading sin Device válido \\
2. Verificar cascade delete de Usuario → Dispositivos → Lecturas \\
3. Probar constraint de un solo cultivo activo por usuario \\
4. Verificar constraint de un solo dispositivo activo por usuario \\ \hline 
\textbf{Resultado esperado} & Base de datos debe rechazar inserciones inválidas y mantener integridad referencial en todas las operaciones. \\ \hline
\end{tabular}
\caption{Caso de prueba PU-M3.2-01: Verificación de integridad de base de datos}
\label{tab:pu-m3.2-01}
\end{table>

\begin{table}[H]
\centering
\begin{tabular}{|l|p{10cm}|}
\hline
\textbf{Identificador} & PU-M3.2-02 \\ \hline
\textbf{Descripción} & Verificar rendimiento de consultas de datos de sensores \\ \hline
\textbf{Precondiciones} & Datos de prueba en tabla SensorReading \\ \hline
\textbf{Pasos a seguir} & 
1. Ejecutar consulta de últimas 20 lecturas por dispositivo \\
2. Medir tiempo de respuesta \\
3. Verificar que índices estén siendo utilizados \\
4. Probar consultas con filtros de fecha \\ \hline 
\textbf{Resultado esperado} & Consultas deben ejecutarse en menos de 100ms y utilizar índices correctamente para optimización. \\ \hline
\end{tabular}
\caption{Caso de prueba PU-M3.2-02: Verificación de rendimiento de consultas}
\label{tab:pu-m3.2-02}
\end{table}

\begin{table}[H]
\centering
\begin{tabular}{|l|p{10cm}|}
\hline
\textbf{Identificador} & PU-M3.2-03 \\ \hline
\textbf{Descripción} & Verificar almacenamiento y recuperación de alertas \\ \hline
\textbf{Precondiciones} & Tabla Alert configurada correctamente \\ \hline
\textbf{Pasos a seguir} & 
1. Insertar alertas de diferentes tipos (riego, sensor, sistema) \\
2. Verificar filtrado por usuario y estado \\
3. Probar actualización de estado a "resuelto" \\
4. Comprobar ordenación por fecha descendente \\ \hline 
\textbf{Resultado esperado} & Sistema de alertas debe almacenar, filtrar y actualizar correctamente con buena performance. \\ \hline
\end{tabular}
\caption{Caso de prueba PU-M3.2-03: Verificación de sistema de alertas}
\label{tab:pu-m3.2-03}
\end{table>

\subsection{Módulo 4: Interfaz Web (Frontend)}

\subsubsection{Dashboard y Visualización de Datos}

\begin{table}[H]
\centering
\begin{tabular}{|l|p{10cm}|}
\hline
\textbf{Identificador} & PU-M4-01 \\ \hline
\textbf{Descripción} & Verificar visualización en tiempo real de datos de sensores \\ \hline
\textbf{Precondiciones} & Usuario logueado con dispositivo activo y datos de sensores \\ \hline
\textbf{Pasos a seguir} & 
1. Acceder al dashboard principal \\
2. Verificar que se muestren las últimas 20 lecturas \\
3. Comprobar actualización automática cada 2 minutos \\
4. Verificar gráficas de temperatura y humedad \\ \hline 
\textbf{Resultado esperado} & Dashboard debe mostrar datos actualizados, gráficas interactivas y actualizarse automáticamente sin intervención manual. \\ \hline
\end{tabular}
\caption{Caso de prueba PU-M4-01: Verificación de dashboard en tiempo real}
\label{tab:pu-m4-01}
\end{table>

\subsubsection{Gestión de Dispositivos y Cultivos}

\begin{table}[H]
\centering
\begin{tabular}{|l|p{10cm}|}
\hline
\textbf{Identificador} & PU-M4-02 \\ \hline
\textbf{Descripción} & Verificar reactividad al cambiar selección de dispositivos y cultivos \\ \hline
\textbf{Precondiciones} & Usuario con múltiples dispositivos y cultivos registrados \\ \hline
\textbf{Pasos a seguir} & 
1. Cambiar dispositivo activo desde interfaz \\
2. Verificar actualización inmediata del dashboard \\
3. Cambiar cultivo seleccionado \\
4. Comprobar actualización de parámetros de riego \\ \hline 
\textbf{Resultado esperado} & La interfaz debe ser completamente reactiva, actualizando dashboard y configuraciones inmediatamente tras cambios de selección. \\ \hline
\end{tabular}
\caption{Caso de prueba PU-M4-02: Verificación de reactividad de selecciones}
\label{tab:pu-m4-02}
\end{table>

\begin{table}[H]
\centering
\begin{tabular}{|l|p{10cm}|}
\hline
\textbf{Identificador} & PU-M4-03 \\ \hline
\textbf{Descripción} & Verificar activación/desactivación de comunicación de dispositivos \\ \hline
\textbf{Precondiciones} & Dispositivo registrado en la interfaz \\ \hline
\textbf{Pasos a seguir} & 
1. Desactivar comunicación desde toggle en interfaz \\
2. Verificar que datos nuevos no aparezcan en dashboard \\
3. Activar comunicación nuevamente \\
4. Comprobar reanudación de recepción de datos \\ \hline 
\textbf{Resultado esperado} & Toggle de comunicación debe controlar efectivamente la recepción y visualización de datos de sensores. \\ \hline
\end{tabular}
\caption{Caso de prueba PU-M4-03: Verificación de control de comunicación}
\label{tab:pu-m4-03}
\end{table>

\subsubsection{Modos de Riego}

\begin{table}[H]
\centering
\begin{tabular}{|l|p{10cm}|}
\hline
\textbf{Identificador} & PU-M4-04 \\ \hline
\textbf{Descripción} & Verificar funcionamiento completo de modos de riego (Manual, Programado, Automático) \\ \hline
\textbf{Precondiciones} & Dispositivo activo, cultivo seleccionado \\ \hline
\textbf{Pasos a seguir} & 
1. Configurar y ejecutar riego manual \\
2. Programar riego con fecha/hora específica \\
3. Configurar umbrales para modo automático \\
4. Verificar controles de pausa/cancelar en todos los modos \\ \hline 
\textbf{Resultado esperado} & Todos los modos de riego deben funcionar correctamente con sus controles específicos y enviar comandos apropiados al dispositivo. \\ \hline
\end{tabular}
\caption{Caso de prueba PU-M4-04: Verificación de modos de riego}
\label{tab:pu-m4-04}
\end{table>

\subsubsection{Sistema de Alertas}

\begin{table}[H]
\centering
\begin{tabular}{|l|p{10cm}|}
\hline
\textbf{Identificador} & PU-M4-05 \\ \hline
\textbf{Descripción} & Verificar visualización y gestión de alertas en interfaz \\ \hline
\textbf{Precondiciones} & Alertas generadas en el sistema \\ \hline
\textbf{Pasos a seguir} & 
1. Acceder a sección de alertas \\
2. Verificar listado con alertas pendientes y resueltas \\
3. Marcar alerta como resuelta \\
4. Comprobar filtrado por estado y fecha \\ \hline 
\textbf{Resultado esperado} & Sistema de alertas debe mostrar información clara, permitir gestión de estados y mantener historial completo. \\ \hline
\end{tabular}
\caption{Caso de prueba PU-M4-05: Verificación de sistema de alertas}
\label{tab:pu-m4-05}
\end{table>

\subsubsection{Modo Demo}

\begin{table}[H]
\centering
\begin{tabular}{|l|p{10cm}|}
\hline
\textbf{Identificador} & PU-M4-06 \\ \hline
\textbf{Descripción} & Verificar funcionamiento del modo demo para visitantes \\ \hline
\textbf{Precondiciones} & Acceso sin autenticación \\ \hline
\textbf{Pasos a seguir} & 
1. Acceder al modo demo desde página principal \\
2. Verificar datos simulados en dashboard \\
3. Comprobar que controles de riego estén deshabilitados \\
4. Verificar que no se puedan realizar cambios persistentes \\ \hline 
\textbf{Resultado esperado} & Modo demo debe mostrar funcionalidades completas con datos simulados pero sin permitir acciones que afecten el sistema real. \\ \hline
\end{tabular}
\caption{Caso de prueba PU-M4-06: Verificación de modo demo}
\label{tab:pu-m4-06}
\end{table}
