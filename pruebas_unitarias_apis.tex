\subsection{Módulo 3.1: Cloud Computing - Servidor Web (Backend)}

\subsubsection{APIs de Autenticación}

\begin{table}[H]
\centering
\scriptsize
\setlength{\tabcolsep}{3pt}
\renewcommand{\arraystretch}{1.8}
\begin{tabularx}{\linewidth}{|l|>{\raggedright\arraybackslash}X|c|>{\raggedright\arraybackslash}X|c|>{\raggedright\arraybackslash}X|}
\hline
\textbf{ID} & \textbf{Endpoint} & \textbf{Método} & \textbf{Descripción} & \textbf{Auth} & \textbf{Resultado Obtenido} \\ \hline
PU-M3.1-01 & \path{/api/auth/register} & POST & Registrar nuevo usuario & No & 201: Usuario creado, email enviado \\
& & & & & 400: Email duplicado \\ \hline
PU-M3.1-02 & \path{/api/auth/login} & POST & Iniciar sesión & No & 200: Token JWT válido \\
& & & & & 401: Credenciales incorrectas \\ \hline
PU-M3.1-03 & \path{/api/auth/verify/:token} & GET & Verificar cuenta por email & No & 200: Cuenta verificada \\
& & & & & 400: Token inválido o expirado \\ \hline
PU-M3.1-04 & \path{/api/auth/forgot-password} & POST & Solicitar recuperación de contraseña & No & 200: Email reset enviado \\
& & & & & 404: Email no encontrado \\ \hline
PU-M3.1-05 & \path{/api/auth/verify-token/:token} & GET & Verificar token de reset de contraseña & No & 200: Token válido \\
& & & & & 400: Token inválido o expirado \\ \hline
PU-M3.1-06 & \path{/api/auth/update-password/:token} & PUT & Actualizar contraseña con token & No & 200: Contraseña actualizada \\
& & & & & 400: Token inválido \\ \hline
PU-M3.1-07 & \path{/api/auth/change-password/:id} & PUT & Cambiar contraseña (usuario autenticado) & Sí & 200: Contraseña cambiada \\
& & & & & 401: No autorizado \\ \hline
PU-M3.1-08 & \path{/api/auth/user} & GET & Obtener perfil de usuario autenticado & Sí & 200: Datos del usuario \\
& & & & & 401: Token inválido \\ \hline
PU-M3.1-09 & \path{/api/auth/admin} & GET & Verificar permisos de administrador & Sí & 200: Usuario es admin \\
& & & & & 403: No es administrador \\ \hline
PU-M3.1-10 & \path{/api/auth/logout} & POST & Cerrar sesión & Sí & 200: Sesión cerrada \\
& & & & & 401: Token inválido \\ \hline
\end{tabularx}
\caption{Pruebas Unitarias - APIs de Autenticación}
\label{tab:pu-apis-auth}
\end{table}

\subsubsection{APIs de Gestión de Usuarios}

\begin{table}[H]
\centering
\scriptsize
\setlength{\tabcolsep}{3pt}
\renewcommand{\arraystretch}{1.8}
\begin{tabularx}{\linewidth}{|l|>{\raggedright\arraybackslash}X|c|>{\raggedright\arraybackslash}X|c|>{\raggedright\arraybackslash}X|}
\hline
\textbf{ID} & \textbf{Endpoint} & \textbf{Método} & \textbf{Descripción} & \textbf{Auth} & \textbf{Resultado Obtenido} \\ \hline
PU-M3.1-11 & \path{/api/users} & POST & Crear usuario (Admin) & Sí & 201: Usuario creado \\
& & & & & 400: Datos inválidos \\ \hline
PU-M3.1-12 & \path{/api/users} & GET & Obtener todos los usuarios (Admin) & Sí & 200: Lista de usuarios \\
& & & & & 403: No es administrador \\ \hline
PU-M3.1-13 & \path{/api/users/:id} & GET & Obtener usuario por ID & Sí & 200: Datos del usuario \\
& & & & & 404: Usuario no encontrado \\ \hline
PU-M3.1-14 & \path{/api/users/:id} & PUT & Actualizar usuario & Sí & 200: Usuario actualizado \\
& & & & & 404: Usuario no encontrado \\ \hline
PU-M3.1-15 & \path{/api/users/:id} & DELETE & Eliminar usuario & Sí & 200: Usuario eliminado \\
& & & & & 404: Usuario no encontrado \\ \hline
PU-M3.1-16 & \path{/api/users/email/:email} & GET & Obtener usuario por email & Sí & 200: Datos del usuario \\
& & & & & 404: Usuario no encontrado \\ \hline
PU-M3.1-17 & \path{/api/users/profile/current} & GET & Obtener perfil del usuario actual & Sí & 200: Datos del usuario \\
& & & & & 401: No autenticado \\ \hline
PU-M3.1-18 & \path{/api/users/account/delete} & DELETE & Eliminar cuenta propia & Sí & 200: Cuenta eliminada \\
& & & & & 400: Error al eliminar \\ \hline
\end{tabularx}
\caption{Pruebas Unitarias - APIs de Gestión de Usuarios}
\label{tab:pu-apis-users}
\end{table}

\subsubsection{APIs de Gestión de Dispositivos}

\begin{table}[H]
\centering
\scriptsize
\setlength{\tabcolsep}{3pt}
\renewcommand{\arraystretch}{1.8}
\begin{tabularx}{\linewidth}{|l|>{\raggedright\arraybackslash}X|c|>{\raggedright\arraybackslash}X|c|>{\raggedright\arraybackslash}X|}
\hline
\textbf{ID} & \textbf{Endpoint} & \textbf{Método} & \textbf{Descripción} & \textbf{Auth} & \textbf{Resultado Obtenido} \\ \hline
PU-M3.1-19 & \path{/api/devices} & GET & Listar dispositivos usuario & Sí & 200: Lista de dispositivos \\
& & & & & 401: No autenticado \\ \hline
PU-M3.1-20 & \path{/api/devices} & POST & Crear nuevo dispositivo & Sí & 201: Dispositivo creado \\
& & & & & 400: Credenciales TTN inválidas \\ \hline
PU-M3.1-21 & \path{/api/devices/:id} & GET & Obtener dispositivo por ID & Sí & 200: Datos del dispositivo \\
& & & & & 404: Dispositivo no encontrado \\ \hline
PU-M3.1-22 & \path{/api/devices/:id} & PUT & Actualizar dispositivo & Sí & 200: Dispositivo actualizado \\
& & & & & 403: No es propietario \\ \hline
PU-M3.1-23 & \path{/api/devices/:id} & DELETE & Eliminar dispositivo & Sí & 200: Dispositivo eliminado \\
& & & & & 409: Dispositivo en uso \\ \hline
PU-M3.1-24 & \path{/api/devices/:id/activate} & PUT & Activar comunicación & Sí & 200: Comunicación activada \\
& & & & & 400: Solo uno activo por usuario \\ \hline
PU-M3.1-25 & \path{/api/devices/:id/deactivate} & PUT & Desactivar comunicación & Sí & 200: Comunicación desactivada \\
& & & & & 404: Dispositivo no encontrado \\ \hline
PU-M3.1-26 & \path{/api/devices/user/:user_id} & GET & Dispositivos por usuario & Sí & 200: Lista de dispositivos \\
& & & & & 403: Sin permisos \\ \hline
PU-M3.1-27 & \path{/api/devices/:id/latest-reading} & GET & Última lectura del dispositivo & Sí & 200: Última lectura \\
& & & & & 404: Sin lecturas disponibles \\ \hline
PU-M3.1-28 & \path{/api/devices/api-key-copied} & POST & Crear alerta de clave API copiada & Sí & 201: Alerta creada \\
& & & & & 401: No autenticado \\ \hline
\end{tabularx}
\caption{Pruebas Unitarias - APIs de Gestión de Dispositivos}
\label{tab:pu-apis-devices}
\end{table}

\subsubsection{APIs de Gestión de Cultivos}

\begin{table}[H]
\centering
\scriptsize
\setlength{\tabcolsep}{3pt}
\renewcommand{\arraystretch}{1.8}
\begin{tabularx}{\linewidth}{|l|>{\raggedright\arraybackslash}X|c|>{\raggedright\arraybackslash}X|c|>{\raggedright\arraybackslash}X|}
\hline
\textbf{ID} & \textbf{Endpoint} & \textbf{Método} & \textbf{Descripción} & \textbf{Auth} & \textbf{Resultado Obtenido} \\ \hline
PU-M3.1-28 & \path{/api/crops} & GET & Listar cultivos usuario & Sí & 200: Lista de cultivos \\
& & & & & 401: No autenticado \\ \hline
PU-M3.1-29 & \path{/api/crops} & POST & Crear nuevo cultivo & Sí & 201: Cultivo creado \\
& & & & & 400: Parámetros inválidos \\ \hline
PU-M3.1-30 & \path{/api/crops/:id} & GET & Obtener cultivo por ID & Sí & 200: Datos del cultivo \\
& & & & & 404: Cultivo no encontrado \\ \hline
PU-M3.1-31 & \path{/api/crops/:id} & PUT & Actualizar cultivo & Sí & 200: Cultivo actualizado \\
& & & & & 403: No es propietario \\ \hline
PU-M3.1-32 & \path{/api/crops/:id} & DELETE & Eliminar cultivo & Sí & 200: Cultivo eliminado \\
& & & & & 409: Cultivo en uso activo \\ \hline
PU-M3.1-33 & \path{/api/crops/:id/select} & PUT & Seleccionar cultivo activo & Sí & 200: Cultivo seleccionado \\
& & & & & 400: Solo uno activo por usuario \\ \hline
PU-M3.1-34 & \path{/api/crops/:id/deselect} & PUT & Deseleccionar cultivo & Sí & 200: Cultivo deseleccionado \\
& & & & & 404: Cultivo no encontrado \\ \hline
PU-M3.1-35 & \path{/api/crops/user/:user_id} & GET & Cultivos por usuario & Sí & 200: Lista de cultivos \\
& & & & & 403: Sin permisos \\ \hline
PU-M3.1-36 & \path{/api/crops/categories} & GET & Obtener categorías de cultivos & No & 200: Lista de categorías \\
& & & & & 500: Error del servidor \\ \hline
\end{tabularx}
\caption{Pruebas Unitarias - APIs de Gestión de Cultivos}
\label{tab:pu-apis-crops}
\end{table}

\subsubsection{APIs de Control de Riego}

\begin{table}[H]
\centering
\scriptsize
\setlength{\tabcolsep}{3pt}
\renewcommand{\arraystretch}{1.8}
\begin{tabularx}{\linewidth}{|l|>{\raggedright\arraybackslash}X|c|>{\raggedright\arraybackslash}X|c|>{\raggedright\arraybackslash}X|}
\hline
\textbf{ID} & \textbf{Endpoint} & \textbf{Método} & \textbf{Descripción} & \textbf{Auth} & \textbf{Resultado Obtenido} \\ \hline
PU-M3.1-37 & \path{/api/irrigation/manual/start} & POST & Iniciar riego manual & Sí & 200: Riego iniciado, downlink enviado \\
& & & & & 409: Ya hay riego activo \\ \hline
PU-M3.1-38 & \path{/api/irrigation/stop} & POST & Detener riego & Sí & 200: Riego detenido, downlink enviado \\
& & & & & 404: No hay riego activo \\ \hline
PU-M3.1-39 & \path{/api/irrigation/pause} & POST & Pausar riego & Sí & 200: Riego pausado \\
& & & & & 400: Riego no pausable \\ \hline
PU-M3.1-40 & \path{/api/irrigation/resume} & POST & Reanudar riego & Sí & 200: Riego reanudado \\
& & & & & 400: Riego no pausado \\ \hline
PU-M3.1-41 & \path{/api/irrigation/status} & GET & Estado actual riego & Sí & 200: Estado del riego \\
& & & & & 404: Sin riego configurado \\ \hline
PU-M3.1-42 & \path{/api/irrigation/programmed} & POST & Configurar riego programado & Sí & 201: Programación guardada \\
& & & & & 400: Fecha/hora inválida \\ \hline
PU-M3.1-43 & \path{/api/irrigation/automatic} & POST & Configurar riego automático & Sí & 201: Umbrales configurados \\
& & & & & 400: Umbrales inválidos \\ \hline
PU-M3.1-44 & \path{/api/irrigation} & POST & Crear configuración de riego & Sí & 201: Configuración creada \\
& & & & & 400: Datos inválidos \\ \hline
PU-M3.1-45 & \path{/api/irrigation/user/:user_id} & GET & Configuraciones por usuario & Sí & 200: Lista de configuraciones \\
& & & & & 403: Sin permisos \\ \hline
PU-M3.1-46 & \path{/api/irrigation/user/:user_id/active} & GET & Configuraciones activas por usuario & Sí & 200: Configuraciones activas \\
& & & & & 404: Sin configuraciones activas \\ \hline
PU-M3.1-47 & \path{/api/irrigation/:id/activate} & PUT & Activar configuración & Sí & 200: Configuración activada \\
& & & & & 404: Configuración no encontrada \\ \hline
PU-M3.1-48 & \path{/api/irrigation/:id/deactivate} & PUT & Desactivar configuración & Sí & 200: Configuración desactivada \\
& & & & & 404: Configuración no encontrada \\ \hline
PU-M3.1-49 & \path{/api/irrigation/manual/:config_id} & PUT & Actualizar configuración manual & Sí & 200: Configuración actualizada \\
& & & & & 400: Datos inválidos \\ \hline
PU-M3.1-50 & \path{/api/irrigation/programmed/:config_id/cancel} & PUT & Cancelar riego programado & Sí & 200: Riego programado cancelado \\
& & & & & 404: Configuración no encontrada \\ \hline
PU-M3.1-51 & \path{/api/irrigation/pump-activations} & POST & Crear activación de bomba & Sí & 201: Activación creada \\
& & & & & 409: Ya hay activación activa \\ \hline
PU-M3.1-52 & \path{/api/irrigation/pump-activations/user/:user_id} & GET & Activaciones por usuario & Sí & 200: Lista de activaciones \\
& & & & & 403: Sin permisos \\ \hline
PU-M3.1-53 & \path{/api/irrigation/pump-activations/active} & GET & Activaciones activas & Sí & 200: Activaciones activas \\
& & & & & 404: Sin activaciones activas \\ \hline
PU-M3.1-54 & \path{/api/irrigation/pump-activations/:id/pause} & PUT & Pausar activación & Sí & 200: Activación pausada \\
& & & & & 400: No se puede pausar \\ \hline
PU-M3.1-55 & \path{/api/irrigation/pump-activations/:id/resume} & PUT & Reanudar activación & Sí & 200: Activación reanudada \\
& & & & & 400: No está pausada \\ \hline
PU-M3.1-56 & \path{/api/irrigation/pump-activations/:id/complete} & PUT & Completar activación & Sí & 200: Activación completada \\
& & & & & 404: Activación no encontrada \\ \hline
\end{tabularx}
\caption{Pruebas Unitarias - APIs de Control de Riego}
\label{tab:pu-apis-irrigation}
\end{table}

\subsubsection{APIs de Datos de Sensores}

\begin{table}[H]
\centering
\scriptsize
\setlength{\tabcolsep}{3pt}
\renewcommand{\arraystretch}{1.8}
\begin{tabularx}{\linewidth}{|l|>{\raggedright\arraybackslash}X|c|>{\raggedright\arraybackslash}X|c|>{\raggedright\arraybackslash}X|}
\hline
\textbf{ID} & \textbf{Endpoint} & \textbf{Método} & \textbf{Descripción} & \textbf{Auth} & \textbf{Resultado Obtenido} \\ \hline
PU-M3.1-57 & \path{/api/sensor-readings/device/:device_id} & GET & Lecturas por dispositivo & No & 200: Array de lecturas con paginación \\
& & & & & Query: limit, offset & 500: Error interno \\ \hline
PU-M3.1-58 & \path{/api/sensor-readings/date-range/:start_date/:end_date} & GET & Lecturas por rango de fechas & No & 200: Lecturas filtradas por fechas \\
& & & & & Params: device\_id requerido & 400: Fechas inválidas \\ \hline
PU-M3.1-59 & \path{/api/sensor-readings} & POST & Crear lectura de sensor & No & 201: Lectura creada exitosamente \\
& & & & & Body: device\_id, air\_humidity, soil\_humidity, temperature & 400: Datos inválidos o faltantes \\ \hline

PU-M3.1-60 & \path{/api/sensor-readings/device/:device_id/latest} & GET & Última lectura por dispositivo & No & 200: Última lectura del dispositivo \\
& & & & & & 404: Sin lecturas para el dispositivo \\ \hline
PU-M3.1-61 & \path{/api/sensor-readings/user/:user_id/active-device/latest} & GET & Última lectura del dispositivo activo & No & 200: Última lectura con tendencias \\
& & & & & & 404: Sin dispositivo activo o lecturas \\ \hline
\end{tabularx}
\caption{Pruebas Unitarias - APIs de Datos de Sensores (Endpoints Implementados)}
\label{tab:pu-apis-sensors}
\end{table}

\subsubsection{APIs de Gestión de Alertas}

\begin{table}[H]
\centering
\scriptsize
\setlength{\tabcolsep}{3pt}
\renewcommand{\arraystretch}{1.8}
\begin{tabularx}{\linewidth}{|l|>{\raggedright\arraybackslash}X|c|>{\raggedright\arraybackslash}X|c|>{\raggedright\arraybackslash}X|}
\hline
\textbf{ID} & \textbf{Endpoint} & \textbf{Método} & \textbf{Descripción} & \textbf{Auth} & \textbf{Resultado Obtenido} \\ \hline
PU-M3.1-64 & \path{/api/alerts} & GET & Listar alertas usuario & Sí & 200: Lista de alertas \\
& & & & & 401: No autenticado \\ \hline
PU-M3.1-65 & \path{/api/alerts} & POST & Crear nueva alerta & Sí & 201: Alerta creada \\
& & & & & 400: Datos inválidos \\ \hline
PU-M3.1-66 & \path{/api/alerts/:id} & GET & Obtener alerta por ID & Sí & 200: Datos de la alerta \\
& & & & & 404: Alerta no encontrada \\ \hline
PU-M3.1-67 & \path{/api/alerts/:id/resolve} & PUT & Marcar alerta resuelta & Sí & 200: Alerta marcada resuelta \\
& & & & & 403: No es propietario \\ \hline
PU-M3.1-68 & \path{/api/alerts/:id} & DELETE & Eliminar alerta & Sí & 200: Alerta eliminada \\
& & & & & 404: Alerta no encontrada \\ \hline
PU-M3.1-69 & \path{/api/alerts/unresolved} & GET & Alertas no resueltas & Sí & 200: Alertas pendientes \\
& & & & & 404: Sin alertas pendientes \\ \hline
PU-M3.1-70 & \path{/api/alerts/resolve-all} & PUT & Resolver todas alertas & Sí & 200: Todas alertas resueltas \\
& & & & & 404: Sin alertas para resolver \\ \hline
PU-M3.1-71 & \path{/api/alerts/my-alerts} & GET & Obtener alertas del usuario autenticado & Sí & 200: Lista de alertas \\
& & & & & 401: No autenticado \\ \hline
PU-M3.1-72 & \path{/api/alerts/old} & DELETE & Eliminar todas las alertas del usuario & Sí & 200: Alertas eliminadas \\
& & & & & 404: Sin alertas para eliminar \\ \hline
PU-M3.1-73 & \path{/api/alerts/user/:user_id} & GET & Alertas por usuario & Sí & 200: Lista de alertas \\
& & & & & 403: Sin permisos \\ \hline
PU-M3.1-74 & \path{/api/alerts/user/:user_id/unresolved} & GET & Alertas no resueltas por usuario & Sí & 200: Alertas no resueltas \\
& & & & & 404: Sin alertas no resueltas \\ \hline
PU-M3.1-75 & \path{/api/alerts/user/:user_id/resolve-all} & PUT & Resolver todas las alertas de un usuario & Sí & 200: Todas alertas resueltas \\
& & & & & 404: Sin alertas para resolver \\ \hline
PU-M3.1-76 & \path{/api/alerts/user/:user_id/old} & DELETE & Eliminar alertas antiguas de un usuario & Sí & 200: Alertas antiguas eliminadas \\
& & & & & 403: Sin permisos \\ \hline
PU-M3.1-77 & \path{/api/alerts/type/:alert_type} & GET & Alertas por tipo & Sí & 200: Alertas del tipo especificado \\
& & & & & 400: Tipo de alerta inválido \\ \hline
PU-M3.1-78 & \path{/api/alerts/severity/:severity} & GET & Alertas por severidad & Sí & 200: Alertas de la severidad especificada \\
& & & & & 400: Severidad inválida \\ \hline
PU-M3.1-79 & \path{/api/alerts/:id/unresolve} & PUT & Marcar alerta como no resuelta & Sí & 200: Alerta marcada como no resuelta \\
& & & & & 404: Alerta no encontrada \\ \hline
PU-M3.1-80 & \path{/api/alerts/stats/type} & GET & Estadísticas de alertas por tipo & Sí & 200: Estadísticas por tipo \\
& & & & & 404: Sin datos estadísticos \\ \hline
PU-M3.1-81 & \path{/api/alerts/stats/severity} & GET & Estadísticas de alertas por severidad & Sí & 200: Estadísticas por severidad \\
& & & & & 404: Sin datos estadísticos \\ \hline
PU-M3.1-82 & \path{/api/alerts/admin/all} & GET & Obtener todas las alertas del sistema (Admin) & Sí & 200: Todas las alertas del sistema \\
& & & & & 403: No es administrador \\ \hline
PU-M3.1-83 & \path{/api/alerts/admin/all} & DELETE & Eliminar todas las alertas del sistema (Admin) & Sí & 200: Todas las alertas eliminadas \\
& & & & & 403: No es administrador \\ \hline
PU-M3.1-84 & \path{/api/alerts/user-registered} & POST & Crear alerta de registro & Sí & 201: Alerta de registro creada \\
& & & & & 400: Datos inválidos \\ \hline


\end{tabularx}
\caption{Pruebas Unitarias - APIs de Gestión de Alertas}
\label{tab:pu-apis-alerts}
\end{table}

\subsubsection{APIs de Integración TTN}

\begin{table}[H]
\centering
\scriptsize
\setlength{\tabcolsep}{3pt}
\renewcommand{\arraystretch}{1.8}
\begin{tabularx}{\linewidth}{|l|>{\raggedright\arraybackslash}X|c|>{\raggedright\arraybackslash}X|c|>{\raggedright\arraybackslash}X|}
\hline
\textbf{ID} & \textbf{Endpoint} & \textbf{Método} & \textbf{Descripción} & \textbf{Auth} & \textbf{Resultado Obtenido} \\ \hline
PU-M3.1-88 & \path{/api/ttn/uplink} & POST & Webhook TTN uplink & No & 200: Datos procesados y almacenados \\
& & & & & 400: Payload inválido \\ \hline
PU-M3.1-89 & \path{/api/ttn/downlink} & POST & Enviar downlink TTN & Sí & 200: Downlink encolado en TTN \\
& & & & & 500: Error en API TTN \\ \hline
PU-M3.1-90 & \path{/api/ttn/led/on} & POST & Encender LED via downlink & No & 200: Downlink enviado exitosamente \\
& & & & & 500: Error en envío downlink \\ \hline
PU-M3.1-91 & \path{/api/ttn/led/off} & POST & Apagar LED via downlink & No & 200: Downlink enviado exitosamente \\
& & & & & 500: Error en envío downlink \\ \hline
\end{tabularx}
\caption{Pruebas Unitarias - APIs de Integración TTN}
\label{tab:pu-apis-ttn-integration}
\end{table}